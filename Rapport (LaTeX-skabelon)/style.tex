\usepackage[latin1]{inputenc} % Skal passe til editorens indstillinger
\usepackage[danish]{babel} % danske overskrifter
\usepackage[T1]{fontenc} % fonte (output)
\usepackage{lmodern} % vektor fonte
\usepackage{graphicx} % inds�ttelse af billeder
\graphicspath{{./billeder/}} % stivej til bibliotek med figurer
\usepackage{mathtools} % matematik - underst�tter muligheden for at bruge \eqref{}
\usepackage[draft,danish]{fixme}
\usepackage{sistyle}
\usepackage{listings}
	\renewcommand\lstlistingname{Kodeeksempel}
	\renewcommand\lstlistlistingname{Kodeeksempel}
% Inds�t todonotes i margin
\usepackage{todonotes}
\SIdecimalsign{,}
\usepackage{cite}
\usepackage{textcomp}
%\usepackage{geometry}
%\geometry{left=3.3cm,top=3cm,right=2.8cm,bottom=3.5cm}
\usepackage{booktabs,dcolumn,array}
\usepackage{multirow}
\renewcommand\multirowsetup{\centering}
\usepackage{cellspace}
	\addtolength\cellspacetoplimit{4pt}
	\addtolength\cellspacebottomlimit{4pt}
%\renewcommand\baselinestretch{1.6}

%\settocdepth{subsection}
%\setsecnumdepth{subsection}

\addto\captionsdanish{
\renewcommand\contentsname{Indholdsfortegnelse}
\renewcommand\chaptername{Afsnit}
\renewcommand\bibname{Litteraturliste}
\renewcommand\appendixname{Appendiks}
}

\usepackage[plainpages=false,pdfpagelabels,pageanchor=false]{hyperref} % aktive links
\hypersetup{%
  pdfauthor={Gruppe 2},
  pdftitle={Modellering, simulering, analogier og eksperimenter},
  pdfsubject={DDF1}}
\usepackage{memhfixc}% rettelser til hyperref

% -- Vis equations og figurere med chapter nummer f�rst for bedre overskuelighed.
\numberwithin{equation}{chapter}
\numberwithin{figure}{chapter}
\usepackage[hang,small,bf]{caption}

\usepackage{fancyhdr}


%Ops�tning af marginer
\addtolength{\hoffset}{-1.6cm}
\addtolength{\textwidth}{3.8cm}
\addtolength{\voffset}{-1.8cm}
\addtolength{\textheight}{4.8cm}
\topmargin = 0cm

\pagestyle{fancyplain}

%Ops�tninger af Header og Footer
\renewcommand{\chaptermark}[1]{\markboth{\thechapter.\ #1}{}}
\fancyhf{}
\lhead{\bfseries \leftmark}
\rhead{\bfseries \thepage}
\renewcommand{\headrulewidth}{0.5pt}
\renewcommand{\footrulewidth}{0.5pt}
\renewcommand{\headwidth}{15.9cm}
\addtolength{\headheight}{0.5pt}

\fancypagestyle{plain}{\renewcommand{\headrulewidth}{0.5pt}}
% Alle forsider er plain, og kan s�ttes op her.
\fancypagestyle{empty}{\renewcommand{\headrulewidth}{0pt} \renewcommand{\footrulewidth}{0pt} \lhead{} \rhead{}}

\fancypagestyle{lines}{\renewcommand{\headrulewidth}{0.5pt} \lhead{} \rhead{}}
% Siderne med Forord og Indholdsfortegnelse skal ikke have titel og nummerering, dette er opsat her.


%Ops�tning af Chapters
\makeatletter
\def\thickhrulefill{\leavevmode \leaders \hrule height 1ex \hfill \kern \z@}
\def\@makechapterhead#1{\vspace*{-10\p@}{\parindent \z@ {\raggedright \LARGE \bfseries \thechapter \;} \LARGE \bfseries #1}\vskip 10 \p@}
\def\@makeschapterhead#1{\vspace*{-10\p@}{\parindent \z@ {\raggedright \Huge \bfseries #1}\vskip 10 \p@}}

\setlength{\headheight}{14pt}

\usepackage[numbib,nottoc]{tocbibind}
